% This a LaTeX template for a research journal, aimed at 
% being 
% 1. easy to use, so one can simply type the daily entries;
% 2. elegant.
% It was written by Níckolas Alves (alves-nickolas.github.io)

% THIS WAS ORIGINALLY COMPILED WITH LUALATEX, so I suggest going to the Menu (top-left of your screen, if you're on Overleaf) and selecting LuaLaTeX on Settings -> Compiler. I didn't test it on XeLaTeX, but I think it should work fine as well. While the document will still compile on pdfLaTeX, for example, it will not have access to the FiraMath font, and hence math text will look weird (it will be typeset in LaTeX's standard Computer Modern Math). If you don't plan on using mathematics at all, then there is a reasonable chance pdfLaTeX will do just fine
\documentclass[a4paper, 11pt, oneside]{researchjournal} % I wrote the design using a4paper, 11pt, oneside, but feel free to change

%\logo{} can be used to add a small decoration to the top of the cover page. My original idea was to put an \insergraphics command in it and load, e.g., the university logo or something

\usepackage{hyperref}
\author{Shad Boswell\\ % you can use double bars to add lines to the author decoration on the main page
University of Utah}

% colors are customizable using xcolor's (https://ctan.org/pkg/xcolor) \definecolor
\definecolor{ChapterBackground}{HTML}{f5f4f2} %colors to use on chapters
\definecolor{ChapterForeground}{HTML}{e93820} %colors to use on chapters
\definecolor{DayColor}{HTML}{df2d16} %colors to use on newdays and daybibs
\definecolor{CoverBackground}{HTML}{f5f4f2} %cover background
\definecolor{CoverForeground}{HTML}{e93820} %cover letters
\definecolor{LinkColor}{HTML}{eb5a00} %color for links

\begin{document} % this will automatically generate a simple cover
\newday{2023-6-11} My research today was split between reading d2l.ai (\cite{d2l.ai}) and watching Andrej Katpathy's video \cite{AndrejKarpathyMicrogradVideo}
\\

Notes on backpropogation and neural networks: 
\begin{enumerate}
    \item Neural networks are just a mathematical expression which can be drawn out using a tree. This tree helps to visualize the expression and understand how backpropogation works
    \item Backpropogation involves starting with the result of an expression and working backwards to see how each parameter to the equation influences the total outcome. It's a "recursive application of backpropogation through the computational tree"
    \item A loss function is a way to tell how well your model is performing. In machine learning, we are trying to minimize our loss as much as possible.
    \item Neurons in a neural network take all the summation of the weights * inputs, add a bias and pass the result to what is known as an "activation function" or "squashing function". 
    \item Squashing functions are typically a sigmoid or tanh function and map the real numbers to the range [0,1]
    
\end{enumerate}
    

\newday{2023-06-12} 

Finishing Andrej Karpathy's video 
(\cite{AndrejKarpathyMicrogradVideo}). Notes: 
\\

\begin{enumerate}
    \item Practice backpropogation. Rules for backpropogation: 
    \begin{enumerate}
        \item Backpropgation through addition disperses the gradient to all of the child nodes
        \item Backpropogation through multiplication takes the the other children's data, multiplies them, and then multiplies by the output gradient
        \item Generally, we take the local derivative of the function and multiply by the output gradient
    \end{enumerate}
    \item Learn the subtleties of Python. The "\_\_rmul\_\_" function serves as fallback for the "\_\_mul\_\_" function and swaps the order of the operands 
    \item Practice using things that you have already written. Example: Andrej implemented subtraction as the addition of a negation. Andrej had already written the addition operation so he just need to write the negation operation which is just a multiplication by -1. Andrej had already written multiplication so this was trivial.  
    \item Pararphrase: "The level at which you implement your operations is up to you... All that matters is that we have some kind of inputs and some kind of outputs and that the output is a function of the input... As long as we can forward pass and backward pass, then the design is up to you." - \cite{AndrejKarpathyMicrogradVideo}
    \item Using PyTorch and Tensors like Micrograd:
    \begin{enumerate}
        \item Need to specify that tensors need a gradient with: n.requires\_grad = True
        \item Tensors have .data.item() which strips out the value in the Tensor
        \item Tensors also have .grad.item() attribute similar to Value.grad in micrograd
        \item The big deal with PyTorch is the efficiency and our ability to take advantage of parallelism with Tensors
    \end{enumerate}
    \item Building a Nerual Net. A neural net is a specific class of mathematical expression.
    \begin{enumerate}
        \item We start by building a Neuron class and then initializing it with some weights and a bias. 
    \end{enumerate}
    \item Loss is a single number that represents the overall performance of our model. In essence it is the difference between the actual output values and the predicted values. Thus, we want to minimize our loss so that our model performs how we expect and accomplishes the task at hand. 
    \item Common loss function is the mean-squared-error function: $$\frac{1}{n}\sum_{i=1}^{n}\left( Y_{i} - \hat{Y_{i}}\right)^2$$ where $Y_i$ is the target values and $\hat{Y_i}$ is the observed values. Why MSE? The further we are away, the greater the loss would be. This effect is exaggerated by squaring the difference between the values. Any difference less than one will also result in a smaller loss than difference which rewards small loss. In a perfect world, our target and our observed value would be the same which would yield 0 loss. 
    \item Do not mess with inputs to neural nets only weights and biases. The total number of parameters for our neural net is the total number of weights + the total number of biases. We can find this by taking the sum of products between layers (weights) and adding the number of neurons in the inner layers + the outer layer (biases)
    \item Modifying parameters: 
    \begin{enumerate}
        \item If a gradient is negative, you increase the parameter 
        \item If a gradient is positive, you decrease the parameter
        \item We can simply just multiply our step size by the -gradient
        \item Andrej states that our gradient vectors point in the direction of increasing loss, so we want to go in the opposite direction. 
        \item Then you just iterate this process until loss is really low (low enough for whatever your standards are). This is called \textbf{gradient descent}!
        \item Be careful, you may overstep! Similar to Newton's method -> You will bounce back and forth between having close observed values and far away observed values. This is called learning rate. We need to find the step size to be just right. You can either take way too long, or explode your loss. 
    \end{enumerate}
    \item Training loops:
    \begin{enumerate}
        \item First do forward pass, then backward pass so we have accurate gradients. Then update. This is the fundamental idea of gradient descent!
        \item \textbf{MAKE SURE YOU FLUSH THE GRADIENT BEFORE BACKWARD PASS!!}
        \item Learning rate decay: Slowing down the learning rate with more iterations (Scaling the learning rate as a function of the number of iterations) allows you to focus on the finer details as the network learns more and more.  
    \end{enumerate}
    \item Summary: 
    \begin{enumerate}
        \item Neural nets: Mathematical expression for a forward pass, followed by a loss function. We then do a backward pass to get the gradient so we can tune each of the parameters to decrease the loss locally. 
        \item When the loss is low, the network is performing how we would like. 
        \item ChatGPT essentially takes a series of words and attempts to predict the next word in the sequence. When you train this on the internet, the network has really amazing emergent properties and has billions of parameters. 
    \end{enumerate}
\end{enumerate}

Reading d2l.ai staring at section 3.1: Regression

Meeting with Ganesh
\begin{enumerate}
    \item You need to share Tensors. That is the name of the game with machine learning. 
    \item groq inc paper - Software-defined tensor streaming multiprocessor for large-scale machine learning. They have their own chips and they essentially want to get rid of the idea of kernels.
    \item NaN in the GPU: Do not have a great status. In the CPU you can use a specific bit pattern to quiet the NaN or make it signal. Read "The Secret life of NaN"
    \item Attacks to floating point computation
    \begin{enumerate}
        \item Injection exceptions in the input to see how far it goes
        \item Lowering precision for some stages
        \item Alterting the inputs to consume a larger dynamic range 
        \item Changing hyper parameters
        \item Changing how the libraries are sparsified
    \end{enumerate}
    How do we find and locate these vulnerabilities? What is the least number of monitoring layers to ensure security of the network?
    \item Look at FPTuner
\end{enumerate}

Watching Andrej Karpathy's video \cite{AndrejKarpathyMakemoreVideo}
\begin{enumerate}
    \item Makemore is designed to makemore of whatever you feed it. 
    \item Makemore is a character level language model. It treats every single line of input as an example and then each as sequences of characters. This is the level on which we are building out makemore. 
    \item With character level language models, we need to be careful of all of the examples packed into a single word. For example, consider the word "isabella". The examples embedded in this word are:
    \begin{enumerate}
        \item "i" will come first
        \item "s" is likely to follow an "i"
        \item "a" is likely to follow "is" 
        \item and so on...
        \item until we get to the end where the word where "a" is likely to follow "isabell"
        \item There is one last example and that is that the word is likely to end after "isabella"
        \item With long words, we have a \textbf{TON} of information
    \end{enumerate}
    \item A bigram is where we try to predict the next letter by only looking at the current letter. 
    \item The zip function in python is great! It takes two iterators together and pairs the iterations together, truncating the longer iterator to meet the shorter one. 
    \item What are the statistics involved in a bigram? The simplest way is just through counting. 
    \begin{enumerate}
        \item Let's keep a dictionary to keep track of the counts of tuples. How? Make a tuple of the characters, get its value from the dictionary (default to 0) and add 1 to it. Then restore it. 
        \item It's actually going to be more convenient to store these counts in 2d arrays where the rows are the first character in the bigram and the columns are the second character in the bigram. We will use PyTorch's Tensor
        \item Tip: Use setbuilder notation to build lookup tables! 
        \item Tip 2: plt.imshow() is pretty cool for plotting 2d arrays
    \end{enumerate}
    \item Now that we have the counts, we can just sample our data repeatedly to create a new word. We can do this by creating probabilities for the starting characters. This is done by normalizing the array of starting characters. 
    \item We then use Python's multinomial function to feed it a probability distribution which it will use to pick a number. The numbers than it can select from are the same as the indices of the distribution. Andrej also used a Python generator object to provide a dterministic way of sampling the probability distribution. (I think was for teaching purposes only and maybe not advised? I will have to look into generators even more, but I think they always pick the same if you give them a specific seed.) 
    \item All in all, bigram language models are pretty terrible. However, they still do something reasonable. If we compare the results to a model that is completely untrained, it performs far better. The words are actually word like where the untrained model just spits out garbage. 
    \item Once you are finished writing a model, be sure to look for inefficiencies that can be solved easily. (Remember Kopta: "The \#1 rule of optimization is: Don't"). The optimization that Andrej did was to factor out the probability calculation from the loop. Rather, we would like to create a 27x27 matrix of probabilities that we can read from. Ideally, we would like each rows probability to be calculated in parallel. 
    \item We can also sum along dimensions with Tensors. For example, with a 2d Tensor, summing along dimension, or axis, 0 will sum the values in the columns to create a 1 x N Tensor. On the other hand, summing along dimension, or axis, 1 will sum the values in a row to create an N X 1 Tensor. This is really helpful because we can then take advantage of Broadcasting! Wooo!
    \item Andrej's tips: READ THROUGH BROADCASTING SEMANTICS! Treat it with respect and take your time. 
    \item In place operations have the potential to be faster. Use them when you can.
\end{enumerate}

\newday{2023-06-13}

I spent most of the day reading from d2l.ai Chapter 3. See the "d2l.ai Notes" Overleaf for updated notes. 
\\

Andrej Karpahty Video: \cite{AndrejKarpathyMakemoreVideo}
\begin{enumerate}
    \item Likelihood estimation. Your model assigns probabilities to every prediction that it makes. If you take your training set and look at the probabilities for each step, they should be close to 1 if your model is performing well. The higher these probabilities are, the better that your model is performing. This is because it is accurately predicting what it needs to predict or what it was trained to predict. The likelihood of a model is the all of the probabilities in your training set multiplied together. This gives an indication of how well your model can predict the entire data set. Since these are less than 0, you often end up with a really small number. Thus, most people work with the log of the likelihood. The log of the likelihood approaches 0 as your likelihood approaches 1 and $-\inf$ as your likelihood approaches 0. This is mostly for convenience: $log(a*b*c) = log(a)+log(b)+log(c)$
    \item How high can log likelihood get? 0 when all probs are 1, $-\infty$ when all probs are 0. This is counter-intuitive with respect to the loss function so we will work with the negative log likelihood instead. This just flips the log around the x-axis. Thus, the lowest you can get is 0 which is the most accurate and the highest you can get is $\infty$ which is the least accurate. 
    \item We want to find the parameters that minimize the negative log likelihood (maximize the likelihood). The average negative log likelihood summarizes your model. 
    \item Model Smoothing: Add some fake counts to everything to smooth out the model. We do not necessarily want to get rid of every single probability. As long as we keep the model smoothing small, our model will still think that the things that were previously 0 are unlikely but will not return an infinite loss. 
    \item Now lets create a neural network for our bigram language model. Pretty much the same approach but with weights, biases, and backpropogation through our loss function. Steps:
    \begin{enumerate}
        \item Create training set of bigrams. This will be made up of two lists: the inputs and labels (targets). Both will be integers to our lookup table. We will then create tensors from these. Example: "emma" has 5 examples for our model: [.e, em, mm, ma, a.]. Thus, our training data will be: [0, 5, 13, 13, 1] and [5, 13, 13, 1, 0]
        \item Tangent: Use torch.tensor not torch.Tensor. torch.Tensor will set dtype to float32 no matter what the dtype actually is. torch.tensor will infer the dtype from what you feed in. 
        \item Tangent: You cannot just plug an integer index into a neural net. We need to pass in a vector. (It really would not make sense to pass an integer index into a neural net. This would skew the weights and the activation of the receiving neuron.) Thus, we introduce one hot encoding. This is a common way of encoding integers into a neural net. Example: For the integer 13, we create a 1X13 vector of all 0's besides the 13th one is a 1. When we plug numbers into neural nets, we want them to be floating type so they can take on a range of values.  
        \item Initialize the weights using randn. We want to do this randomly so we don't mess anything up. 
        \item Multiply our encoding by weights. (Matrix Multiply '@'). This is super efficient!!! We can feed in all 5 of our examples and simultaneously compute the activation of each neuron for each example. 
        \item We want 27 neurons in our first layer. So we change the shape of our weights matrix to 27x27. These neurons are do not have a bias nor a nonlinearity. \item We are not going to have any other layers. The simplest neural net. A single layer. 
        \item what should these 27 outputs be? We are trying to produce a probability distribution for each character so we can predict what the next character will be. Right now, we just have some positive and negative numbers. Probabilities are positive and they sum to one. Thus, these cannot come from a neural net. 
        \item How do we interpret the output? The outputs are log(counts) (logits) so we just exponentiate them. Then we can use these to calculate probability. Thus, creating a distribution which we can sample to pick the next letter??
        \item We then get $n$ rows for every one of our $n$ examples. Each row is a distribution. The neural nets assignment for how likely each one of the 27 numbers is to come next.  
        \item Softmax: Taking the logits, exponentiating them, and then normalizing to get a probability. Very common layer in neural nets. A way of taking the outputs of a neural net layer and creating a probability distribution from them. The outputs can be negative, positive, any number really. but the result of the softmax is a vector which is totally positive and sums to 1. 
        \item Each operation is easy to back propagate through and differentiable. Thus, we can train the model with gradient descent. 
    \end{enumerate}

    \item How to optimize a neural net. (\textbf{DO NOT GUESS AND CHECK. According to Andrej: "This is amateur hour"})
    The big deal is that we have a loss function! We can minimize this loss through back propagation and gradient descent!!! Booyah! Remember the algorithm for gradient descent: 
    \begin{enumerate}
        \item Forward pass
        \item Calculate loss
        \item Zero-gradients 
        \item Back propagate
        \item Update parameters in the opposite direction of the gradient
        \item Repeat until "done" (you choose/good enough accuracy)
    \end{enumerate}
    \item We are using the negative log likelihood because we are doing classification and not regression. Reminder: Look into this more.
    \item Calculating our loss:
    \begin{enumerate}
        \item Pluck out the probabilities corresponding to our targets. 
        \item Take the log of each 
        \item Take the average
        \item Negate the result
        \item This is our loss function for classification
    \end{enumerate}
    \item Zero out the gradients. You can just set them to None. (This is more efficient)
    \item Now we back propagate. Reminder: We need to set requires\_grad=True when we intialize our weights. We cannot back propagate if we do not do this. Now, just call, loss.backward(). AUTOMAGIC!
    \item Now, W.grad tells us how each weight to each neuron influences the total loss. Thus, we now have a direction for updating the parameters. Reminder: We want to update in the opposite direction of the gradient. this looks like W += -$\eta$ * W.grad where $\eta$ is the learning rate. 
    \item When we achieve low loss, this means the network is assigning high probability to the characters that should come next, low probability to the characters that should not come next, and everything in between
    \item Tangent: Optimization in this case just means getting your neural network to perform better/be more accurate. 
    \item \textbf{IMPORTANT: }We actually end up approaching the same loss value that we got when we just counted everything. We are just approaching the same problem from a different perspective. This is because we are not taking in anymore data than we had prior. We are still analyzing bigrams: looking at only the previous character to try and predict the next character. Note, this is only because the bigram statistics are very straightforward and the problem is not complicated. Gradient descent learning is much more flexible and can accommodate much more complex problems. (Still make sure you analyze the complexity of what you are trying to solve and choose the approach that suits the problem best). 
    \item How are neural nets more flexible? We can use the same basis and fundamentals to set up our network. Rather than just feeding in one character to one layer, we can feed in multiple and have multiple layers. 
    \item What are the fundamentals of a neural net? 
    \begin{enumerate}
        \item The output will always be logits
        \item We take these logits through a softmax
        \item We take the output of the soft max to calculate our loss (negative log)
        \item We then backpropogate through our loss
    \end{enumerate}
    \item The only thing that will really change is how we do our forward pass. 
    \item The one-hot vectors multiplied by the weights: What is actually happening is the one hot just plucks out the corresponding weight. 
    \item Trying to incentivize W to be closer to 0 is equivalent to smoothing. The more that we send each W to 0, the more uniform our distribution gets? This introduces what is known as regularization. We can add regularization term to our loss calculation to try and incentivize W to go to 0, while at the same time trying to get the probabilities to match what we want the model to predict. Note: Look more into regulariztation later! We have to choose a regularization strength. You can think of regularization as a gravitational force and the weights desire to reach a probability that we want as something growing upward. The regularization sort of resists the changing of the weights. If it is too high, then we accumulate way too much loss and the weights cannot reach a value that results in an accurate model. With the right regularization, the distribution gets smoothed slightly, but the weights are still able to achieve their desired values, resulting in a model that is accurate, not-buggy, and realistic?
\end{enumerate}

\newday{2023-06-14}

Harvey's Notebook \#4 (A lot of these notes will be direct paraphrases of Harve's notebook. You can find the notebook here: \cite{HarveyDamTREUGit}

\begin{enumerate}
    \item Cross-Correlation: A convolutional layer followed by an addition. The cross correlation of vectors $w$ and $x$ is $w \star x = z \in \mathbb{R}^{|x|-|w|+1}$. For example, suppose $w = [1,2,3]$ and $x = ]1,2,3,4,5]$. Then $w \star x = [1 * 1 + 2 * 2 + 3 * 3, 1 * 2 + 2 * 3 + 3 * 4, 1 * 3 + 2 * 4 + 3 * 5]$
    \item Convolution: In ML this most often means cross-correlation. Generally a convolution is: an integral that expresses the amount of overlap of one function $g$ as it is shifted over another function $f$. In other words, we are shifting the function $g$ over $f$ to extract some crucial information. In cross-correlation, we are shifting $w$ over $x$ to see how $w$ changes $x$. Kernel choice in convolution matters greatly and it all depends on the information we want to extract from $x$
    \item How are convolutional layers beneficial with respect to simple linear transformations? Convolutional layers allow us to pick out certain features from the image. If we know an image that we are trying to classify has very specific features, we can try to define a convolutional layer that will pull out those specific features. This allows for easier classification and labeling. We can just look at the data after applying a convolutional layer and say "Hey that looks like (or does not) this...". 
    \item As a side note, convolutions can help us detect edges in images, filter images, sharpen images, and much more
    \item Convolutional network layers have many kernels of different shapes and strides (how many pixels to move the kernel over after each $i$ or $j$ step. The kernels are learnable, so their elements serve as parameters in the neural network
    \item Paraphrase: "Each kernel learns to detect a certain feature in the input, all combining to distill the information that we want from the input. Earlier layers will detect more simple shapes, while later layers will detect more sophisticate things such as shapes or faces."
 \end{enumerate}
\end{document} % this will automatically generate a references chapter if you cited any references throughout the text